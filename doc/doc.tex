\documentclass[13pt,a4paper]{report}

\usepackage[T1]{fontenc}
\usepackage[romanian]{babel}
\usepackage{fontspec}
\usepackage[colorlinks]{hyperref}

\addto\captionsromanian{\renewcommand{\abstractname}{Abstract}}

\begin{document}

\title{Money Stock}
\author{Barbu Paul - Gheorghe\\
Colegiul Național "Gheorghe Lazăr" - Sibiu\\
\texttt{paul.barbu@cnglsibiu.ro}}
\date{}
\maketitle

\begin{abstract}
Documentația programului Money Stock pentru Concursul de Informatică Aplicată
2012.
Acest program a fost creat plecând de la tema „Plus și Minus” având în vedere
fluctuațiile cursului valutar.
Aplicația este scrisă în C\# și folosește sursa on-line
\href{http://bnr.ro}{www.bnr.ro} pentru a
obține cursul valutar pentru monedele disponibile.
\end{abstract}

\section{Scopul aplicației}
Scopul acestei aplicații este de a oferi utilizatorului o interfață simplă și
intuitivă pentru a putea schimba sume imaginare de bani în diferite monede.
Pe lângă funcționalitatea de bază, aplicația oferă posibilitatea de afișare a
cursurilor sub forma de grafic, o metodă intuitivă pentru studierea evoluției
investițiilor.

\section{Seturile de date}
MoneyStock folosește cursurile valutare puse la dispoziție de BNR prin
intermediul fișierelor XML.
La fiecare pornire aplicația descarcă și parsează aceste feed-uri populând o bază
de date embedded.

Doar la prima pornire aplicația va descărca feed-ul din anul precedent \\ (calculat în
program), de exemplu
\href{http://bnr.ro/files/xml/years/nbrfxrates2011.xml}{feed-ul din anul 2011}, 
și va introduce aceste date în baza de date.

Apoi se vor prelua seturile de \href{http://bnr.ro/nbrfxrates10days.xml}{10
zile} și \href{http://bnr.ro/nbrfxrates.xml}{setul curent de date} lucru care se
va întâmpla la orice pornire ulterioară cu setul curent de date.
Setul de zece zile va fi preluat doar dacă datele locale sunt mai
vechi de zece zile.

\section{Baza de date}
Baza de date folosită este \texttt{Microsoft SQL Server Compact 3.5} fiind una de
tip embedded nu necesită existența unui server, putând fi creată local pe orice
calculator.
Schema este integrată în program, tabelele fiind create dinamic din cod în
funcție de datele primite de la BNR.

Fiecare monedă are un tabel propriu după cum urmează:

\begin{center}
\begin{tabular}{| c | c |}
    \multicolumn{2}{ c }{nume monedă} \\ \hline
    rata de schimb & data cursului \\
    \hline
\end{tabular}
\end{center}

Cele două câmpuri reprezentând rata de schimb a monedei și data
calendaristică când aceasta e valabilă.

\section{Cursul valutar}
Principala funcționalitate oferită de Money Stock este cea de conversie a unor
sume de bani între monezi, aceasta oferind și posibilitatea adăugării TVA-ului la
suma curentă.
De asemenea este posibilă alegerea unei date calendaristice, caz în care ratele
de schimb se vor modifica reflectându-le pe cele de la data
selectată.

%TODO add images and explain them with a numbered list

\section{Statistică} %TODO check if it's named statistici
Când utilizatorul selectează tab-ul „Statistici” are posibilitatea de a vedea o
reprezentare a datelor sub forma de grafic.
Acest mod îi permite cu ușurință să facă comparații și să observe cum a evoluat
cursul valutar pe o perioadă de timp.

Perioada predefinită este de două săptămâni, iar monedele pre-selectate sunt
euro și dolar-ul american, acestea putând fi modificate cu ușurință.

%TODO explain the zoom
Când pe grafic sunt reprezentate mai multe monede pe o perioadă lungă de timp,
acesta își pierde din acuratețe, acest lucru fiind contracarat de funcția
\texttt{zoom}.
Zoom-ul poate fi activat selectând cu click stânga o regiune a graficului
aceasta urmând să fie mărită și acuratețea reprezentării datelor să crească.

Pentru revenire la nivelul normal de zoom se poate da un click dreapta pe
grafic. Pentru a reveni un singur nivel de zoom trebuie folosit butonul 
%TODO add image 
de lângă barele de derulare.

\section{Funcții suplimentare}
%mărimea dinamica a form-ului
%update DB

\section{Validare și securitate}
%the input box on the main tab
%the validity of the dates on both tabs
%checking for internet connection

\section{Cerințe}
%DLL
%.net fw 3.5

\end{document}
